% Options for packages loaded elsewhere
\PassOptionsToPackage{unicode}{hyperref}
\PassOptionsToPackage{hyphens}{url}
%
\documentclass[
  ignorenonframetext,
]{beamer}
\usepackage{pgfpages}
\setbeamertemplate{caption}[numbered]
\setbeamertemplate{caption label separator}{: }
\setbeamercolor{caption name}{fg=normal text.fg}
\beamertemplatenavigationsymbolsempty
% Prevent slide breaks in the middle of a paragraph
\widowpenalties 1 10000
\raggedbottom
\setbeamertemplate{part page}{
  \centering
  \begin{beamercolorbox}[sep=16pt,center]{part title}
    \usebeamerfont{part title}\insertpart\par
  \end{beamercolorbox}
}
\setbeamertemplate{section page}{
  \centering
  \begin{beamercolorbox}[sep=12pt,center]{part title}
    \usebeamerfont{section title}\insertsection\par
  \end{beamercolorbox}
}
\setbeamertemplate{subsection page}{
  \centering
  \begin{beamercolorbox}[sep=8pt,center]{part title}
    \usebeamerfont{subsection title}\insertsubsection\par
  \end{beamercolorbox}
}
\AtBeginPart{
  \frame{\partpage}
}
\AtBeginSection{
  \ifbibliography
  \else
    \frame{\sectionpage}
  \fi
}
\AtBeginSubsection{
  \frame{\subsectionpage}
}
\usepackage{amsmath,amssymb}
\usepackage{lmodern}
\usepackage{iftex}
\ifPDFTeX
  \usepackage[T1]{fontenc}
  \usepackage[utf8]{inputenc}
  \usepackage{textcomp} % provide euro and other symbols
\else % if luatex or xetex
  \usepackage{unicode-math}
  \defaultfontfeatures{Scale=MatchLowercase}
  \defaultfontfeatures[\rmfamily]{Ligatures=TeX,Scale=1}
\fi
% Use upquote if available, for straight quotes in verbatim environments
\IfFileExists{upquote.sty}{\usepackage{upquote}}{}
\IfFileExists{microtype.sty}{% use microtype if available
  \usepackage[]{microtype}
  \UseMicrotypeSet[protrusion]{basicmath} % disable protrusion for tt fonts
}{}
\makeatletter
\@ifundefined{KOMAClassName}{% if non-KOMA class
  \IfFileExists{parskip.sty}{%
    \usepackage{parskip}
  }{% else
    \setlength{\parindent}{0pt}
    \setlength{\parskip}{6pt plus 2pt minus 1pt}}
}{% if KOMA class
  \KOMAoptions{parskip=half}}
\makeatother
\usepackage{xcolor}
\newif\ifbibliography
\usepackage{graphicx}
\makeatletter
\def\maxwidth{\ifdim\Gin@nat@width>\linewidth\linewidth\else\Gin@nat@width\fi}
\def\maxheight{\ifdim\Gin@nat@height>\textheight\textheight\else\Gin@nat@height\fi}
\makeatother
% Scale images if necessary, so that they will not overflow the page
% margins by default, and it is still possible to overwrite the defaults
% using explicit options in \includegraphics[width, height, ...]{}
\setkeys{Gin}{width=\maxwidth,height=\maxheight,keepaspectratio}
% Set default figure placement to htbp
\makeatletter
\def\fps@figure{htbp}
\makeatother
\setlength{\emergencystretch}{3em} % prevent overfull lines
\providecommand{\tightlist}{%
  \setlength{\itemsep}{0pt}\setlength{\parskip}{0pt}}
\setcounter{secnumdepth}{-\maxdimen} % remove section numbering
\ifLuaTeX
  \usepackage{selnolig}  % disable illegal ligatures
\fi
\IfFileExists{bookmark.sty}{\usepackage{bookmark}}{\usepackage{hyperref}}
\IfFileExists{xurl.sty}{\usepackage{xurl}}{} % add URL line breaks if available
\urlstyle{same} % disable monospaced font for URLs
\hypersetup{
  pdftitle={Demand Prediction For Bike Sharing Systems},
  pdfauthor={Ahmed Elsabbagh, Hagar Tarek, Loay Wael, Taha Ashri, Ahmed Saber, and Osama},
  hidelinks,
  pdfcreator={LaTeX via pandoc}}

\title{Demand Prediction For Bike Sharing Systems}
\author{Ahmed Elsabbagh, Hagar Tarek, Loay Wael, Taha Ashri, Ahmed
Saber, and Osama}
\date{}

\begin{document}
\frame{\titlepage}

\hypertarget{project-description}{%
\section{Project Description}\label{project-description}}

\begin{frame}{Introduction}
\protect\hypertarget{introduction}{}
\begin{itemize}
\item
  Bike Sharing Systems (BSS) is a very widely used method of
  transportation for many people worldwide.
\item
  They have several advantages:

  \begin{itemize}
  \tightlist
  \item
    Non-polluting.
  \item
    Circumvents traffic congestion.
  \item
    Excellent for last-mile connections.
  \item
    Have a convenient payment system.
  \end{itemize}
\end{itemize}

\includegraphics[width=2.91667in,height=\textheight]{images/bike_sharing_sweden.jpg}
\includegraphics[width=2.91667in,height=\textheight]{images/Hangzhou_bike_sharing_station.jpg}
\includegraphics[width=2.91667in,height=\textheight]{images/hello_bike.jpg}
\end{frame}

\begin{frame}{Motivation}
\protect\hypertarget{motivation}{}
\includegraphics[width=3.95833in,height=\textheight]{images/moving_truck.jpg}

\begin{itemize}
\tightlist
\item
  BSSs use self-serve bikes locked to docking stations.
\item
  Bikes are unlocked, used for a trip, then left at the closest station
  to their destination.
\item
  Bikes need to be relocated from low demand to high demand stations.
\end{itemize}
\end{frame}

\begin{frame}{Problem Definition}
\protect\hypertarget{problem-definition}{}
\begin{itemize}
\tightlist
\item
  The relocation process could be costly and time consuming.
\item
  Stations with high demand cannot be identified in advance.
\item
  Stations with high return rate cannot be identified in advance.
\item
  Relocation may not be important if return rate is high enough to fill
  the demand.
\end{itemize}
\end{frame}

\begin{frame}{Objectives}
\protect\hypertarget{objectives}{}
The objective is to create multivariate regression models capable of
estimating the following

\begin{itemize}
\tightlist
\item
  \textbf{Demand prediction:} Number of bikes needed at each individual
  station.
\item
  \textbf{Return prediction:} Number of bikes returned at each
  individual station.
\end{itemize}
\end{frame}

\hypertarget{background}{%
\section{Background}\label{background}}

\begin{frame}{Demand Prediction Models-Models}
\protect\hypertarget{demand-prediction-models-models}{}
\begin{itemize}
\tightlist
\item
  \textbf{Univariate Time Series: }

  \begin{itemize}
  \tightlist
  \item
    Autoregressive moving Average (ARMA)
  \item
    Autoregressive Integrated Moving Average (ARIMA)
  \end{itemize}
\item
  \textbf{Multivariate Time Series: }

  \begin{itemize}
  \tightlist
  \item
    Deep Learning with LSTM
  \item
    Random Forest.
  \item
    Boosting Algorithms.
  \end{itemize}
\end{itemize}
\end{frame}

\begin{frame}{Demand Prediction Models-Features}
\protect\hypertarget{demand-prediction-models-features}{}
\begin{itemize}
\item
  \textbf{Demographics:}

  \begin{itemize}
  \tightlist
  \item
    Wealth.
  \item
    Average age.
  \end{itemize}
\item
  \textbf{Community}
\item
  \textbf{Weather}
\item
  \textbf{Sliding Window Statistics:}

  \begin{itemize}
  \tightlist
  \item
    Standard deviations.
  \item
    Averages.
  \end{itemize}
\end{itemize}

NB: Weather is especially important. BIXI BSS in Montreal is not
operational in unfavorable weather, typically in mid-November to April.
\end{frame}

\hypertarget{dataset}{%
\section{Dataset}\label{dataset}}

\begin{frame}{Lyft/Baywheels}
\protect\hypertarget{lyftbaywheels}{}
The data was provided by Lyft, the owner of the BSS company Baywheels
(formally GoBike) operating in San Francisco Bay Area.

\begin{itemize}
\tightlist
\item
  The data is available as far as 2017 before Lyft's acquisition of
  GoBike.
\item
  The data used however is limited to 2021 and 2022 because:

  \begin{itemize}
  \tightlist
  \item
    The disruption caused by the 2020 COVID pandemic.
  \item
    Potential business growth gap between 2019 and 2021.
  \end{itemize}
\end{itemize}
\end{frame}

\begin{frame}{Weather Data}
\protect\hypertarget{weather-data}{}
\begin{itemize}
\tightlist
\item
  Weather data is very relevant in determining the demand.
\item
  The weather data for San Francisco Bay Area is provided by Meteostat
  through a Python API.
\end{itemize}

\includegraphics[width=0.5\textwidth,height=\textheight]{images/meteostat_example.png}
\end{frame}

\begin{frame}{Dataset Description}
\protect\hypertarget{dataset-description}{}
\begin{block}{Size:}
\protect\hypertarget{size}{}
\begin{itemize}
\tightlist
\item
  \textbf{Original Number of Trips:} 4744199 trips .
\item
  \textbf{Final Number of Trips:} 3855197 trips.
\item
  \textbf{Number Of Stations}: 535
\item
  \textbf{Period:} From January 2021 to December 2022
\end{itemize}
\end{block}

\begin{block}{Relevant Columns:}
\protect\hypertarget{relevant-columns}{}
\begin{itemize}
\tightlist
\item
  Station Name (start/end)
\item
  Time (start/end)
\item
  Coordinates (start/end)
\item
  Ride ID
\end{itemize}
\end{block}
\end{frame}

\hypertarget{cleaning-process}{%
\section{Cleaning Process}\label{cleaning-process}}

\begin{frame}{Cleaning Process}
\protect\hypertarget{cleaning-process-1}{}
\begin{itemize}
\item
  \textbf{Station Standardization:}

  \begin{itemize}
  \tightlist
  \item
    Stations in a particular street don't always have the same
    coordinates.
  \item
    Many stations don't have a standardized name or ID, making them
    unidentifiable.
  \item
    \textbf{Solution To Reduce Data Loss}:

    \begin{itemize}
    \tightlist
    \item
      Use a single coordinate for any identifiable stations.
    \item
      Approximate the closest standard station to any trip.
    \item
      If the closest station is less than 500 meters away, keep the
      trip, otherwise, drop.
    \end{itemize}
  \end{itemize}
\end{itemize}
\end{frame}

\begin{frame}{Cleaning Process}
\protect\hypertarget{cleaning-process-2}{}
\begin{itemize}
\tightlist
\item
  \textbf{Same Station Trip:}

  \begin{itemize}
  \tightlist
  \item
    Several trips usually take place in several minutes, with the end
    station being the same as start stations.
  \item
    This could be a result of users trying out the system or changed
    minds.
  \item
    To prevent redundancy, any same station trip with duration less than
    4 minutes will be removed.
  \end{itemize}
\end{itemize}
\end{frame}

\begin{frame}{Cleaning Process}
\protect\hypertarget{cleaning-process-3}{}
\begin{itemize}
\tightlist
\item
  \textbf{Clustering}:

  \begin{itemize}
  \tightlist
  \item
    Helps reduce weather data size by:

    \begin{itemize}
    \tightlist
    \item
      Approximating areas closest to each other.
    \item
      Approximating the weather conditions for each cluster.
    \end{itemize}
  \item
    Separates the areas the BSS serves.
  \end{itemize}
\end{itemize}
\end{frame}

\begin{frame}{Cleaning Process}
\protect\hypertarget{cleaning-process-4}{}
\end{frame}

\begin{frame}{Demand and Returns}
\protect\hypertarget{demand-and-returns}{}
\textbf{Definitions}

\begin{itemize}
\tightlist
\item
  Demand: Number of trips starting at a particular station.
\item
  Returns: Number of trips ending at a particular station.
\item
  ``Returns'' does not indicate actual number of bikes available, but it
  indicates how many bikes finished their trips there, which when
  compared with demand should provide a good idea about the available
  bikes.
\item
  Therefore, demand will be the main focus, while returns will be taken
  as secondary.
\end{itemize}
\end{frame}

\begin{frame}{Demand}
\protect\hypertarget{demand}{}
\end{frame}

\begin{frame}{Demand and Returns}
\protect\hypertarget{demand-and-returns-1}{}
\end{frame}

\begin{frame}{Demand Spread}
\protect\hypertarget{demand-spread}{}
\end{frame}

\begin{frame}{Demand Spread}
\protect\hypertarget{demand-spread-1}{}
\end{frame}

\hypertarget{results}{%
\section{Results}\label{results}}

\begin{frame}{Testing Method}
\protect\hypertarget{testing-method}{}
\includegraphics[width=0.5\textwidth,height=\textheight]{images/test_split.png}
\end{frame}

\begin{frame}{Daily Demand Prediction}
\protect\hypertarget{daily-demand-prediction}{}
\end{frame}

\begin{frame}{Hourly Demand Prediction}
\protect\hypertarget{hourly-demand-prediction}{}
\end{frame}

\begin{frame}{Daily Demand Prediction Example Station}
\protect\hypertarget{daily-demand-prediction-example-station}{}
\end{frame}

\begin{frame}{Hourly Demand Prediction Example Station}
\protect\hypertarget{hourly-demand-prediction-example-station}{}
\end{frame}

\begin{frame}{RMSE}
\protect\hypertarget{rmse}{}
There are several criteria that can calculate the RMSE:

\begin{itemize}
\tightlist
\item
  Hourly Predictions: 1.4093 Bikes Error.
\item
  Overall Daily Predictions: 4.4659 Bikes Daily
\item
  Mean RMSE Hourly Per Station: 3.6386 Bikes hourly
\end{itemize}
\end{frame}

\hypertarget{references}{%
\section{References}\label{references}}

\begin{frame}{}
\protect\hypertarget{section}{}
\begin{itemize}
\tightlist
\item
  Susan Shaheen, Stacey Guzman, and Hua Zhang. Bikesharing in europe,
  the americas, and asia. Transportation Research Record, pages
  159--167, 1 2010.
\item
  Andreas Nikiforiadis, Katerina Chrysostomou, and Georgia Aifadopoulou.
  Exploring travelers' characteristics affecting their intention to
  shift to bike-sharing systems due to a sophisticated mobile app.
  Algorithms, 12:264, 12 2019.
\item
  Jung-Hoon Cho, Young-Hyun Seo, and Dong-Kyu Kim. Efficiency comparison
  of public bike-sharing repositioning strategies based on predicted
  demand patterns. Transportation Research Record: Journal of the
  Trans-portation Research Board, 2675:104--118, 11 2021.
\item
  Aliasghar Mehdizadeh Dastjerdi and Catherine Morency. Bike-sharing
  demand prediction at community level under covid-19 using deep
  learning. Sensors, 22:1060, 1 2022.
\item
  Bike share in the san francisco bay area --- bay wheels --- lyft.
\item
  Ahmed Ghanem, Hesham A. Rakha, and Leanna House. Modeling bike
  availability in a bike-sharing system using machine learning. pages
  374--378. IEEE, 6 2017.
\item
  Andreas Kaltenbrunner, Rodrigo Meza, Jens Grivolla, Joan Codina, and
  Rafael Banchs. Urban cycles and mobility patterns: Exploring and
  predicting trends in a bicycle-based public transport system.
  Pervasive and Mobile Computing, 6:455--466, 8 2010.
\item
  Alvaro Lozano, Juan De Paz, Gabriel Villarrubia Gonz ́alez, Daniel
  Iglesia, and Javier Bajo. Multi-agent system for demand prediction and
  trip visualization in bike sharing systems. Applied Sciences, 8:67, 1
  2018.
\end{itemize}
\end{frame}

\begin{frame}{}
\protect\hypertarget{section-1}{}
\begin{itemize}
\tightlist
\item
  R. Alexander Rixey. Station-level forecasting of bikesharing
  ridership. Transportation Research Record: Journal of the
  Transportation Research Board, 2387:46--55, 1 2013.
\item
  Young-Hyun Seo, Sangwon Yoon, Dong-Kyu Kim, Seung-Young Kho, and
  Jaemin Hwang. Predicting demand for a bike-sharing system with station
  activity based on random forest. Proceedings of the Institution of
  Civil Engineers - Municipal Engineer, 174:97--107, 6 2021
\end{itemize}
\end{frame}

\hypertarget{thank-you}{%
\section{Thank you!}\label{thank-you}}

\end{document}
